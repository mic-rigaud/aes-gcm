\documentclass[a4paper, 11pt, oneside, oldfontcommands]{memoir}

%%%%% Packages %%%%%
\usepackage{lmodern}
\usepackage{palatino}
\usepackage[T1]{fontenc}
\usepackage[utf8]{inputenc}
\usepackage[french]{babel}


%%%%%%%%%%%%%%%%%%%%  PACKAGE SECONDAIRE

%\usepackage{amstext,amsmath,amssymb,amsfonts} % package math
%\usepackage{multirow,colortbl}	% to use multirow and ?
%\usepackage{xspace,varioref}
\usepackage[linktoc=all, hidelinks]{hyperref}			% permet d'utiliser les liens hyper textes
\usepackage{float}				% permet d ajouter d autre fonction au floatant
%\usepackage{wrapfig}			% permet d avoir des image avec texte coulant a cote
%\usepackage{fancyhdr}			% permet d inserer des choses en haut et en bas de chaque page
\usepackage{microtype}			% permet d ameliorer l apparence du texte
\usepackage[explicit]{titlesec}	% permet de modifier les titres
\usepackage{graphicx}			% permet d utiliser les graphiques
\graphicspath{{./img/}}		% to say where are image
%\usepackage{eso-pic} 			% to put figure in the background
\usepackage[svgnames]{xcolor}	% permet d avoir plus de 300 couleur predefini
%\usepackage{array}				% permet d ajouter des option dans les tableaux
%\usepackage{listings}			% permet d ajouter des ligne de code
%\usepackage{tikz}				% to draw figure
%\usepackage{appendix}			% permet de faire les index
%\usepackage{makeidx}			% permet de creer les index
%\usepackage{fancyvrb}			% to use Verbatim
%\usepackage{framed}				% permet de faire des environnement cadre
%\usepackage{fancybox}			% permet de realiser les cadres
\usepackage{titletoc}			% permet de modifier les titres
%\usepackage{caption}
\usepackage[a4paper, top=2cm, bottom=2cm]{geometry}
\usepackage{frbib}                      %permet d avoir une biblio francaise
\usepackage[babel=true]{csquotes}


\usepackage{graphicx}
\RequirePackage{pageGardeEnsta}	% permet d avoir la page de garde ensta

%\setcounter{secnumdepth}{2}		% permet d'augmenter la numerotation
%\setcounter{tocdepth}{2}		% permet d'augmenter la numerotation

%%%%%%%%%%%%%%%%%%  DEFINITION DES BOITES
\newcounter{rem}[chapter]

\newcommand{\remarque}[1]{\stepcounter{rem}\noindent\fcolorbox{OliveDrab}{white}{\parbox{\textwidth}{\textcolor{OliveDrab}{
\textbf{Remarque~\thechapter.\therem~:}}\\#1}}}

\newcounter{th}[chapter]

\newcommand{\theoreme}[2]{\noindent\fcolorbox{FireBrick}{white}{\stepcounter{th}
\parbox{\textwidth}{\textbf{\textcolor{FireBrick}{Théorème~\thechapter.\theth~:}}{\hfill \textit{#1}}\\#2}}}

\newcommand{\attention}[1]{\noindent\fcolorbox{white}{white}{\parbox{\textwidth}{\textcolor{FireBrick}{
\textbf{Attention !}}\\\textit{#1}\\}}}
%%%%%%%%%%%%%%%%%%%%%%%%%%%%%%%%%%%%%%%%%%%%%%%%%%%%%%%%%%%%%%%%%%%%%%%%%


%% INDEX %%%%%%%%%%%%%%%%%%%%%%%%%%%%%%%%%%%%%%%%%%%%%%%%%%%%
\makeindex

%%%%% Useful macros %%%%%
\newcommand{\latinloc}[1]{\ifx\undefined\lncs\relax\emph{#1}\else\textrm{#1}\fi\xspace}
\newcommand{\etc}{\latinloc{etc}}
\newcommand{\eg}{\latinloc{e.g.}}
\newcommand{\ie}{\latinloc{i.e.}}
\newcommand{\cad}{c'est-à-dire }
\newcommand{\st}{\ensuremath{\text{\xspace s.t.\xspace}}}
\newcommand{\aes}{AES-GCM }

%%%% Definition des couleur %%%%

\newcommand\couleurb[1]{\textcolor{SteelBlue}{#1}}
\newcommand\couleurr[1]{\textcolor{DarkRed}{#1}}


%% number page style style %%%%%%%%%%%%%%%%%%%%%%%%%%%%%%%%%%%%%%%%%%%%%%%%%%%%%%

\pagestyle{plain}
%\pagestyle{empty}
%\pagestyle{headings}
%\pagestyle{myheadings}



%% chapters style %%%%%%%%%%%%%%%%%%%%%%%%%%%%%%%%%%%%%%%%%%%%%%%%%%%%%%
%% You may try several styles (see more in the memoir manual).

%\chapterstyle{veelo}
%\chapterstyle{chappell}
%\chapterstyle{ell}
%\chapterstyle{ger}
%\chapterstyle{pedersen}
%\chapterstyle{verville}
\chapterstyle{madsen}
%\chapterstyle{thatcher}


%%%%% Report Title %%%%%
\title{AES-GCM}
\author{\textsc{Rigaud Michaël} et \textsc{Badier Charlie}}
\date{\today}
\doctype{UV4.8}
\promo{promo 2017}
\etablissement{\textsc{Ensta} Bretagne\\2, rue François Verny\\
  29806 \textsc{Brest} cedex\\\textsc{France}\\Tel +33 (0)2 98 34 88 00\\ \url{www.ensta-bretagne.fr}}
\logoEcole{\includegraphics[height=4.2cm]{logo_ENSTA_Bretagne_Vertical_CMJN}}



%%%%%%%%%%%%%%%%%% DEBUT DU DOCUMENT
\begin{document}

\maketitle
\thispagestyle{empty}
\newpage

\tableofcontents


%%%%%%%%%%%%%%%%% INTRODUCTION

\chapter*{Introduction}
\addcontentsline{toc}{chapter}{Introduction}

GCM ou Galois Counter Mode est un mode d'opération de chiffrement par bloc en cryptographie symétrique. C'est un algorithme de chiffrement authentifié qui garanti l'intégrité et l'authenticité des données. Lors des opérations que nous verrons plus loin, cet algorithme demande de chiffrer avec un autre algorithme de chiffrement. D'après la norme IEEE 802.1AE, on utilise l'algorithme AES (Advanced Encryption Standard). On appelle donc cet algorithme \aes.

Dans ce rapport, nous expliquerons dans un premier temps le fonctionnement de \aes ainsi que de ses autres modes. Puis nous le comparerons à d'autres algorithmes semblables en termes de complexité. Enfin, nous essayerons de voir quels sont les principaux vecteurs d'attaques de cet algorithme dans les applications usuelles.


\newpage	  
%%%%%%%%%%%%%%%%%%%%%%%%

\chapter{Différents modes de fonctionnement cryptographique}
\label{chap:différents modes}

Il existe plusieurs modes de fonctionnement cryptographiques qui peuvent être mis en oeuvre avec l'algorithme de chiffrement AES. Nous verrons dans un premier temps les principaux modes de fonctionnement.

\section{ECB}

Le mode ECB (Electronic codebook ou dictionnaire de code) est le plus simple. Il consiste à diviser le message à chiffrer en blocs qui vont être chiffrés indépendament les uns des autres. Pour le déchiffrement on procédera de la même manière en découpant le texte chiffré en blocs et en décryptant les blocs indépendament les uns des autres.

\begin{figure}[!h]
  \centering
  \includegraphics[width=\textwidth]{fonctionnement-ECB}
  \caption{schéma ECB \cite{wiki}}
  \label{schema ECB}
\end{figure}

Ce mode présente les avantages du chiffrement par flots, est pré-calculable et est parallélisable. Il offre la possibilité de déchiffrer une zone quelconque du texte chiffré et ainsi de déchiffrer une partie seulement des données.

Cependant ce mode possède un défaut considérable: deux blocs de texte clair seront chiffrés de la même manière, car il n'y a pas de randomisation. Ce défaut rend le mode ECB vulnérable aux attaques par dictionnaire et à l'analyse fréquentielle. En effet pour une clef donnée, on pourra générer un dictionnaire avec les correspondances entre les clairs et le chiffrés, permettant ainsi de retrouver le texte clair. Pour ces raisons l'utilisation de ce mode est fortement déconseillé.

\section{CBC}

Avec le mode CBC (Cipher Block Chainning ou Enchaînement des blocs), on applique à chaque bloc de texte clair un "XOR" (ou exclusif) avec le bloc chiffré précedent. Ainsi chaque bloc chiffré dépend des blocs traités auparavavant. Pour le premier bloc il faut fournir un vecteur d'initialisation.

\begin{figure}[!h]
  \centering
  \includegraphics[width=\textwidth]{fonctionnement-CBC}
  \caption{schema CBC - Chiffrement \cite{wiki}}
  \label{schema CBC - Chiffrement}
\end{figure}

Ce mode présente possède les avantages du chiffrement par flots, et il offre également la possibilité de déchiffrer une zone quelconque du texte chiffré. Cependant un des inconvénients est que le chiffrement est séquentiel ( \cad il ne peut pas être parallélisé).

Pour le déchiffrement, on passe le premier bloc crypté dans le déchiffrement de bloc et on effectue un "XOR" avec le vecteur d'initialisation IV. Dans le cas où le vecteur d'initialisation est incorrect seul le premier bloc crypté sera impossible à décrypter. En effet à chaque bloc on applique un "XOR" avec le chiffré du bloc précédent, et pas le texte clair. Ainsi on peut retrouver un bloc de texte clair uniquement à partir du bloc crypté précédent, ce qui permet ainsi la parallélisation de la décryption. 

\begin{figure}[!h]
  \centering
  \includegraphics[width=\textwidth]{fonctionnement-CBC_de}
  \caption{schema CBC - Déchiffrement}
  \label{schema CBC - Déchiffrement}
\end{figure}




\section{CFB}
Le mode CFB (Cipher FeedBack ou Chiffrement à rétroaction) est similaire au mode CBC. Tout comme le CBC, ce mode permet de déchiffrer n'importe quelle zone du chiffré. Cependant, comme le CBC, le chiffrement est séquentiel, il ne peut donc pas être parallèlisé. Le déchiffrement est similaire au CBC et peut, quant à lui, être parallélisé. 

\begin{figure}[!h]
  \centering
  \includegraphics[width=\textwidth]{fonctionnement-CFB}
  \caption{schema CFB - Chiffrement}
  \label{schema CFB - Chiffrement}
\end{figure}

\begin{figure}[!h]
  \centering
  \includegraphics[width=\textwidth]{fonctionnement-CFB_de}
  \caption{schema CFB - Déchiffrement}
  \label{schema CFB - Déchiffrement}
\end{figure}

\section{OFB}
Le mode OFB (Output FeedBack) est une variante du mode CFB. En effet, au lieu d'utiliser un bloc chiffré pour chiffrer le suivant, le mode OFB va utiliser le chiffré du vecteur d'initialisation. S'il s'agit du bloc N, alors celui-ci sera chiffré avec le vecteur d'initialisation chiffré N fois. Le décryptage est très proche du CFB, il faut juste prendre le déchiffré du vecteur d'initialisation.


\begin{figure}[!h]
  \centering
  \includegraphics[width=\textwidth]{fonctionnement-OFB}
  \caption{schema OFB - Chiffrement}
  \label{schema OFB - Chiffrement}
\end{figure}

\begin{figure}[!h]
  \centering
  \includegraphics[width=\textwidth]{fonctionnement-OFB_de}
  \caption{schema OFB - Déhiffrement}
  \label{schema OFB - Déchiffrement}
\end{figure}

\section{CTR}
Comme le mode OFB, le mode CTR permet le chiffrement par flot et est pré-calculable. De plus il offre un accès aléatoire aux données, est parallélisable et n'utilise que la fonction de chiffrement. 

\begin{figure}[!h]
  \centering
  \includegraphics[width=\textwidth]{fonctionnement-CTR}
  \caption{schema CTR - Chiffrement}
  \label{schema CTR - Chiffrement}
\end{figure}


%%% Local Variables: 
%%% mode: latex
%%% TeX-master: "rapport_de_base"
%%% End: 

\chapter{Et le reste du monde...}
\label{chap:autres}

Pas du tout sur...

\section{OCB}

\section{IAPM}

\section{EAX}

\section{CWC}

\section{CCM}





%%% Local Variables: 
%%% mode: latex
%%% TeX-master: "rapport_de_base"
%%% End: 

\chapter{Comparaison GCM - CCM - OCB}

Lors de nos recherches nous avons constaté que les algorithmes fournissant à la foi la confidentialité, l'authenticité et l'intégrité sont: GCM, CCM, OCB, CWC, EAX et APM.

Mais parmi eux seulement trois sortent du lot GCM, OCB et CCM. C'est pourquoi nous avons décidé d'écarter les autres algorithmes et de seulement comparer ces trois.


\section{CCM}

Comme son nom le suggère le mode CCM combine le mode CTR et le mode CBC-MAC. C'est deux modes "primitifs" sont combinés pour authentifier les données puis les encrypter. CBC-MAC permet dans un premier temps d'obtenir un tag d'authentification du message clair. Puis le message et le tag sont chiffrés en utilisant le mode CTR.



Le vecteur d'initialisation (IV) doit est être choisi avec précaution car il ne doit jamais être utilisé plus d'un fois par clef. En effet le mode CCM est un dérivé du mode CTR.



Une idée essentielle est que la même clé de cryptage peut être utilisée à la fois pour l'authentification et l'encryptage, à condition que les valeurs de comptage utilisées dans le cryptage ne rentrent pas en collision avec le vecteur d'initialisation (IV) utilisé pour l'authentification.


\begin{figure}[!h]
  \centering
  \includegraphics[width=\textwidth]{fonctionnement-CBC_MAC}
  \caption{Création du tag d'authentification avec CBC-MAC}
  \label{Création du tag d'authentification avec CBC-MAC}
\end{figure}

\begin{figure}[!h]
  \centering
  \includegraphics[width=\textwidth]{fonctionnement-CTR}
  \caption{Chiffrement du tag et du message avec CTR}
  \label{Chiffrement du tag et du message avec CTR}
\end{figure}


Contrairement à \aes, la génération du tag d'authentification (via CBC-MAC) est fait à partir du message en clair et non pas à partir du message encrypté. Au niveau performance, le mode CCM va utiliser de manière général plus de calcul que \aes. En effet, dans \aes l'authentification et l'encryption ne font appel qu'une seule fois à l'algorithme AES par bloc tandis que, le mode CCM va utiliser une première fois le chiffrement AES pour l'encryption puis une seconde fois pour générer le tag d'authentification. De plus, le mode CCM ne peut pas être parallélisé.


\newpage


\section{OCB}
Le mode OCB est lui aussi conçu pour fournir à la fois l'authentification et la confidentialité. OCB (Offset CodeBook) est basé sur le mode ECB avec l'utilisation d'un vecteur d'initialisation. Pour l'authentification il faut d'abord effectuer un checksum du message clair, ce checksum est ensuite encrypter comme un block du message clair. 

\begin{figure}[!h]
  \centering
  \includegraphics[width=\textwidth]{fonctionnement-OCB}
  \caption{Fonctionnement du mode OCB}
  \label{Fonctionnement du mode OCB}
\end{figure}

AES-OCB est un mode de fonctionnement qui est soumis à deux brevet au Etat-Unis qui empêchent son utilisation dans toute application commerciale ou gouvernementale au Etat-Unis. 
De plus Niels Ferguson a montrer l'existence d'attaque sur le mode OCB qui limite l'envoie de données à 64GB par clef. D'un point de vue performance, le mode OCB semble plus rapide que \aes car il ne nécessite pas l'implémentassions des bloc de multiplications dans l'espace de Galois.


\section{Comparaison des performances}

\subsection{comparaison extrait d'une étude}


Pour obtenir les performances de ces modes nous nous sommes appuyés sur le travail de Ted KROVETZ et Phillip ROGAWAY \cite{compa}.

Sur la figure \ref{fig:compa}, il est possible d'observer la rapidité de modes OCB, CCM et GCM sur des architectures différentes. Sur l'axe des abscisses on peut observer la taille du message en bytes, et sur les ordonnées le nombre de cycle par byte.

On remarque que sur la totalité des architectures testées OCB est plus performant que ces rivales.

\begin{figure}[!h]
  \centering
  \includegraphics[width=\textwidth]{comparaison2}
  \caption{Performance empirique sur différents architectures\cite{compa}}
  \label{fig:compa}
\end{figure}

Cette étude confirme ce que nous avions révélé précédemment, \cad que OCB est plus performant que \aes.%Grâce à cette étude, nous pouvons en conclure que l'algorithme le plus performant est OCB en terme de rapidité. Ceci s'explique comme nous avons pu le remarquer précédemment sur la simplicité de l'algorithme.


\subsection{Comparaison réalisé sur nos machines personnelles}



Sur nos machines (Asus N76VB avec Ubuntu 14.04) nous avons également essayé de vérifier ces performances. Pour cela nous avons d'abord utilisé le site \url{http://www.faux-texte.com/lorem-ipsum-20.htm} pour générer un faux texte pseudo-aléatoire de 6.8M\footnote{Il est possible de visualiser le fichier que nous avons utiliser sur notre github \url{https://github.com/magichal/aes-gcm}} . Ensuite, nous avons exécuter la commande suivante dans notre terminale linux:

\begin{lstlisting}
  time openssl enc -e -in "fichier_test" -aes-256-gcm -out "fichier_encrypt" -k "toto"
\end{lstlisting}

  Voici la sortie que nous avons pu observer:

  \begin{lstlisting}
   openssl enc -e -in "fichier_test" -aes-256-gcm -out "fichier_encrypt" -k   0,02s user 0,01s system 98% cpu 0,027 total
  \end{lstlisting}

Donc notre ordinateur a mis 0.02s pour chiffrer notre fichier.

~\\

Malheureusement nous ne pourrons pas comparer cette performance au mode OCB et CCM puisqu'ils ne sont pas implémentés dans openssl. Néanmoins, nous pouvons constater la rapidité de l'opération qui a été réalisé. Nous supposons que dans une utilisation courante cette rapidité est plus que suffisante pour un utilisateur lambda.

De plus au cours de notre réflexion nous avons envisager de proposer une amélioration à \aes en remplaçant AES par un autre algorithme de chiffrement. Nous pensons néanmoins que cela sort du cadre de notre étude c'est pourquoi il est possible de trouver en annexe à la page \pageref{anexe1} les alternatives à AES.




%%% Local Variables: 
%%% mode: latex
%%% TeX-master: "rapport_de_base"
%%% End: 


%%%% CONCLUSION %%%%%%%%%

\chapter*{Conclusion}
\addcontentsline{toc}{chapter}{Conclusion}

Face à cette étude nous pouvons conclure que \aes est un algorithme très puissant et rapide qui permet d'assurer la confidentialité, l'intégrité et l'authenticité des données.

Il est évidemment a recommander par rapport au autres mode d'AES qui n'assure pas l'authenticité. Mais nous avons également vu qu'il fallait privilégier dans l'algorithme GCM l'utilisation de AES par rapport à Treefish ou Salsa20 qui sont bien moins utilisé.

Enfin nous avons constaté que l'utilisation de GCM est bien plus performante que d'autre algorithme comme OCB ou CCM pour réalisé l'authenticité et l'intégrité.


\newpage

%%%% ANNEXE %%%%%%%%%%%%

\part*{Annexe}
\appendix
\nocite{*}
\input{Partie4}
%\input{annexe_}
\newpage
 \listoffigures
 \printindex
 \bibliographystyle{frplain}
  \bibliography{biblio}

\end{document}
%%%%%%%%%%%%%%%%% FIN DU DOCUMENT
